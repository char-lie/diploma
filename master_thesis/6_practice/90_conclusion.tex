\chapterConclusion

Проведено теоритичний аналіз проблем
у використанні запропонованих в минулому розділі методів.
Виявилося, що найбільш коректний з теоретичної точки зору
метод реконструкції, що використовує метод Монте-Карло,
практично неможливо використовувати на даний момент.
Баєсова стратегія з інтервальною функцією витрат
може бути реалізована досить скоро.
Стало зрозуміло, що сучасні методи базуються на
оцінці найбільшої правдоподібності,
тому що цей підхід не тільки дає достатню точність,
але є одним з небагатьох, який доцільно використовувати.

Наведені результати демонструють базові можливості
просторової реконструкції людського обличчя за портретом.
Використані методи працюють та дають реалістичні результати.
Проте необхідні подальші дослідження,
щоб виявити слабкі та сильні сторони різних підходів.
