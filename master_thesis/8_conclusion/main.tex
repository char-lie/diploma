\chapter*{Висновки}
\addcontentsline{toc}{chapter}{Висновки}

В результаті виконання роботи вдалося
розробити оптимізований метод реконструкції просторової конфігурації
людського обличчя в системах біометричної ідентифікації.

Попередні роботи приголомшують своїми результатами.
Їх аналіз допоміг задати правильні питання та шукати на них відповідь.
Також вони містять хитрощі, до яких треба вдаватися,
щоб обходити певні складні ситуації.

За допомогою статистичної теорії розпізнавання образів
було зроблено коректну постановку та теоретичний розв'язок поставленої задачі.
Добре відомі чисельні методи вказали,
як обходити обчислювальні складнощі,
що виникають при практичному розв'язку задачі.

Було реалізовано демонстративне програмне забезпечення,
що дає гарну реконструкцію просторової конфігурації
людського обличчя за певних умов.
Реалізації деяких методів вказали на необхідність
різного роду еврістичних підходів
та неможливість практичного застосування деяких теоретично вірних міркувань.
