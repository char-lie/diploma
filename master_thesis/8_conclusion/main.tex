\chapter*{Висновки}
\addcontentsline{toc}{chapter}{Висновки}

В результаті виконання роботи вдалося
закласти теоретичне підґрунтя для подальшої роботи над задачею
створення оптимізованого методу реконструкції просторової конфігурації
людського обличчя в системах біометричної ідентифікації.

Попередні роботи приголомшують своїми результатами.
Їх аналіз допоміг задати правильні питання та шукати на них відповідь.
Також вони містять хитрощі, до яких треба вдаватися,
щоб обходити певні складні ситуації.

За допомогою статистичної теорії розпізнавання образів
було зроблену коректну постановку та теоретичний розв'язок поставленої задачі.
Перевірені чисельні методи вказали,
як обходити обчислювальні складнощі,
що виникають при практичному вирішенні задачі.

Було реалізовано демонстративне програмне забезпечення,
що дає гарну реконструкцію просторової конфігурації
людського обличчя за певних умов.
