\chapter*{Вступ}
\addcontentsline{toc}{chapter}{Вступ}

\textbf{Актуальність роботи.}
Примітивні системи біометричної ідентифікації особистості за фотознімком
для коректної роботи потребують певних умов
освітлення, положення обличчя на знімку, виразу обличчя
та інших параметрів фотозйомки.
Наразі ці обмеження вдається обходити за допомогою різних підходів,
серед яких є
використання насичених навчальних виборок при машинному навчанні,
розпізнавання опорних точок обличчя (ніс, рот, очі)
та просторова реконструкція обличчя.
Методи для реалізації останнього підходу наразі розвиваються,
різні досліджувачі пропонують свої алгоритми,
які швидше або точніше, ніж у інших.

\textit{Об’єкт дослідження}~---~обличчя.

\textit{Предмет дослідження}~---~ідентифікація особистості за знімком обличчя.

\textbf{Мета дослідження.}
Розробка оптимізованого методу реконструкції просторової конфігурації
людського обличчя в системах біометричної ідентифікації.

Завдання наступні:
\begin{enumerate}
  \item
    Дослідити існуючі методи реконструкції просторової конфігурації обличчя
    за одним або кількома зображеннями;
  \item
    Проаналізувати існуючі методи з теоретичної точки зору.
  \item
    Запропонувати кращий метод.
\end{enumerate}

\textbf{Практичне значення одержаних результатів.}
Реконструкцію тривимірної моделі поверхні обличчя
можна використовувати в системах біометричної ідентифікації,
для реконструкції облич історичних постатей,
у відеоиграх і так далі.
Неточності, що були знайдені в існуючих алгоритмах,
допоможуть побудувати програмне забезпечення,
що робить реконстукцію обличчя точніше за аналоги.
