\chapter*{Вступ}
\addcontentsline{toc}{chapter}{Вступ}

\textbf{Актуальність роботи.}
Примітивні системи біометричної ідентифікації за фотознімком
для коректної роботи потребують певних умов
освітлення, положення обличчя на знімку, виразу обличчя
та інших параметрів фотозйомки.
Наразі ці обмеження вдається обходити за допомогою різних підходів,
серед яких є
використання насичених навчальних виборок при машинному навчанні,
розпізнавання опорних точок обличчя
та просторова реконструкція обличчя.
Методи для реалізації останнього підходу наразі розвиваються,
різні досліджувачі пропонують свої алгоритми,
які швидше або точніше, ніж у інших.

\textit{Об’єкт дослідження}~---~методи біометричної ідентифікації за
знімком обличчя.

\textit{Предмет дослідження}~---~методи реконструкції просторової конфігурації
людського обличчя.

\textbf{Мета дослідження.}
Розробка оптимізованого методу реконструкції просторової конфігурації
людського обличчя в системах біометричної ідентифікації.

Завдання наступні:
\begin{enumerate}
  \item
    дослідити існуючі методи реконструкції просторової конфігурації обличчя
    за одним або кількома зображеннями;
  \item
    проаналізувати існуючі методи з теоретичної точки зору;
  \item
    покращити метод реконструкції просторової конфігурації людського обличчя
    за одним або кількома зображеннями.
\end{enumerate}

\textbf{Практичне значення одержаних результатів.}
Реконструкцію тривимірної моделі поверхні обличчя
можна використовувати в системах біометричної ідентифікації,
для реконструкції облич історичних постатей,
у відеоиграх і так далі.
Неточності, що були знайдені в існуючих алгоритмах,
допоможуть побудувати програмне забезпечення,
що робить реконстукцію обличчя точніше за аналоги.

\textbf{Публiкацiї.}
ХV Всеукраїнська науково-практична конференцiя студентiв,
аспiрантiв та молодих вчених
<<Теоретичнi i прикладнi проблеми фiзики, математики та iнформатики>>.
