\chapter*{Вступ}
\addcontentsline{toc}{chapter}{Вступ}

\textbf{Актуальність роботи.}
Примітивні системи біометричної ідентифікації особистості за фотознімком
для коректної роботи потребують певних умов
освітлення, положення обличчя на знімку, виразу обличчя
та інших параметрів фотозйомки.
Наразі ці обмеження вдається обходити за допомогою різних підходів,
серед яких є
використання насичених навчальних виборок при машинному навчанні,
просторова реконструкція обличчя та
розпізнавання опорних точок обличчя (ніс, рот, очі).
Наступним етапом є подолання таких складно передбачуваних перешкод,
як наявність косметики, аксесуарів (сережки, окуляри),
волосся на обличчі (чубчик, борода, вуса).
Деякі аксесуари можна зняти перед розпізнаванням,
проте бороду та косметику складно прибирати кожен раз лише для того,
щоб бути розпізнаним.

\textit{Об’єкт дослідження} --- обличчя.

\textit{Предмет дослідження} --- ідентифікація особистості за знімком обличчя.

\textbf{Мета дослідження.}
Розробка оптимізованого методу реконструкції просторової конфігурації
людського обличчя в системах біометричної ідентифікації.

Завдання наступні:
\begin{enumerate}
  \item
    Дослідити існуючі методи реконструкції просторової конфігурації обличчя
    за одним або кількома зображеннями;
  \item
    Проаналізувати існуючі методи з теоретичної точки зору.
  \item
    Запропонувати кращий метод.
\end{enumerate}

\textbf{Практичне значення одержаних результатів.}
