\section{Питання щодо постановки задачі}

Перше фундаментальне питання полягає в тому,
навіщо потрібно мінімізувати саме такі функції, як було вказано вище.
В роботі \cite{blanz:romdhani:vetter} вибір функції витрат
\eqref{eq:energy:blanz} мотивується принципом оцінки апостеріорного максимуму.
Проте функція \eqref{eq:energy:face2face} немає підґрунтя окрім того,
що її мінімізація дає блискучі результати.

У формулі \eqref{eq:energy:face2face} є
вагові коефіцієнти $\omega_c$, $\omega_r$ і $\omega_l$,
походження яких не пояснюється.
Також невідомо, чому значення мають бути сами такими,
і чи є вони універсальними або залежать від вхідних даних.

У статтях зазначається,
що енергія є складно влаштованою функцією,
яка має багато локальних мінімумів.
Для подолання цієї складності
автори згадуваних статей пропонують
модифікації класичних алгоритмів
та цільової функції.
Досліджень цих локальних екстремумів не було знайдено:
за якими параметрами, за яких умов і наскільки глибокі локальні мінімуми ---
можливо, ця інформація допомогла б знайти рішення,
яке допоможе відрізнити локальний мінімум від глобального.
