\section{Висновки}

Запропоновані методи досить складно порівняти з практичної точки зору:
для цього треба реалізувати їх на одній мові програмування
та перевірити на одній тестовій виборці на однакових обчислювальних машинах.
Проте було вказано математичні неточності та неясності,
що були знайдені при вивченні даних методів.

Щоб проанализувати доцільність алгоритмів
з точки зору статистичної теорії розпізнавання,
в роботі буде запропоновано кілька постановок задачі,
їх розв'язки та модифікації/спрощення, що дозволять отримати методи,
які були запропоновані у розглянутих публікаціях.
На основі цих даних можна буде проаналізувати якість існуючих алгоритмів
з теоретичної точки зору.
