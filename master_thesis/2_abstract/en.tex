\chapter*{Abstract}

The thesis contains \pageref{LastPage} pages,
\TotalValue{totalfigures} figures
and \total{citenum} references.

Technical progress has reached such performance,
that inverse rendering of human face can be made on an average laptop
for minutes.
Though, existent methods can be improved theoretically and practically.

This study aims to design an optimized method
of human face inverse rendering in biometric identification systems.
This will allow to use correctly reconstructed models
in biometric identification systems.

To perform the study
\begin{itemize}
  \item
    Basel morphable face model was used;
  \item
    bayesian patterns recognition theory was used
    to formulate the problem correctly and analyse existent solutions;
  \item
    numerical methods were used to compute the solution.
\end{itemize}

\MakeUppercase{bayesian strategy, patterns recognition,
 inverse rendering, morphable model,
biometric identification}
