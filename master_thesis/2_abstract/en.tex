\chapter*{Abstract}

Technical progress has reached such performance,
that inverse rendering of human face can be made on an average laptop
for minutes.
Though,
existent methods can be improved theoretically and practically.

This study aims to formulate and solve the recognition problem
in terms of Bayesian patterns recognition theory
introduced by Theodore W. Anderson,
and provide methods, which will help to implement the solution.
This will allow to use correctly reconstructured models
in biometric identification systems.

To perform the study
\begin{enumerate}
  \item
    Basel morphable model was used as a generative model for the face surface;
  \item
    Bayesian patterns recognition theory was used
    to formulate the problem correctly and analyse existent solutions;
  \item
    Numerical methods were used to compute the solution.
\end{enumerate}

\MakeUppercase{Bayesian strategy, patterns recognition,
 inverse rendering, morphable model,
biometric identification}
