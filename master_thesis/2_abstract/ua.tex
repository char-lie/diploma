\chapter*{Реферат}

Технічний прогрес досят таких висот,
що реконструювати просторову конфігурацію людського обличчя за знімком
можливо на звичайному ноутбуці за кілька хвилин.
Тим не менш, існуючі методи можна покращити як теоретично, так і практично.

Метою роботи є постановка та розв'язок задачі розпізнавання
в термінах Баєсової теорії розпізнавання,
що була запропонована Теодором В. Андерсоном,
і вказати методи, які допоможуть реалізувати рішення програмно.
Це допоможе використовувати коректно розпізнані моделі
в системах біометричної ідентифікації.

Для досягнення мети було використано
\begin{enumerate}
  \item
    Модель обличчя, зроблену в Базелівському університеті,
    в якості породжуючої моделі поверхні людського обличчя;
  \item
    Баєсова теорія розпізнавання образів була використана
    для коректної постановки задачі і аналізу існуючих рішень;
  \item
    Чисельні методи були використані для розрахунку рішення.
\end{enumerate}

\MakeUppercase{Баєсова стратегія, розпізнавання образів,
просторова реконструкція, породжуюча модель,
біометрична ідентифікація}
