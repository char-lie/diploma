\chapter*{Реферат}

Дисертація містить \pageref{LastPage} сторінки,
\TotalValue{totalfigures} ілюстрацій
і бібліографію з \total{citenum} найменувань.

Технічний прогрес досяг таких висот,
що реконструювати просторову конфігурацію людського обличчя за знімком
можливо на звичайному ноутбуці за кілька хвилин.
Тим не менш, існуючі методи можна покращити як теоретично, так і практично.

Метою роботи є
розробка оптимізованого методу реконструкції просторової конфігурації
людського обличчя в системах біометричної ідентифікації.
Це допоможе використовувати коректно розпізнані моделі
в системах біометричної ідентифікації.

Для досягнення мети було використано
\begin{itemize}
  \item
    Базельську породжувальну модель поверхні обличчя;
  \item
    баєсову теорію розпізнавання образів
    для коректної постановки задачі і аналізу існуючих рішень;
  \item
    чисельні методи для розрахунку рішення.
\end{itemize}

\MakeUppercase{баєсова стратегія, розпізнавання образів,
просторова реконструкція, породжувальна модель,
біометрична ідентифікація}
