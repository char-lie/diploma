\chapter*{Реферат}

Диссертация содержит \pageref{LastPage} страницы,
\TotalValue{totalfigures} иллюстраций
и список использованной литературы из
\total{citenum} наименований.

Технический прогресс шагнул настолько далеко,
что пространственную реконструкцию человеческого лица по снимку
теперь возможно выполнить на обычном ноутбуке.
Тем не менее,
существующие методы можно улучшить как теоретически, так и практически.

Цель данной работы~---~разработка метода
реконструкции пространственной конфигурации человеческого лица
в системах биометрической идентификации.
Это даст возможность использовать корректно распознанные модели
в системах биометрической идентификации.

Для достижения цели были использованы
\begin{enumerate}
  \item
    Базелевская модель головы
    как порождающая модель поверхности лица;
  \item
    Байесова теория распознавания образов
    для правильной постановки и решения задачи и анализа существующих решений;
  \item
    Численные методы для расчёта решения.
\end{enumerate}

\MakeUppercase{байесова стратегия, распознавание образов,
пространственная реконструкция, порождающая модель,
биометрическая идентификация}
