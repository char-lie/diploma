\chapter*{Реферат}

Технический прогресс шагнул настолько далеко,
что пространственную реконструкцию человеческого лица по фото
теперь возможно выполнить на обычном ноутбуке.
Тем не менее,
существующие методы можно улучшить как теоретически, так и практически.

Цель данной работы --- постановка и решение заадачи распоззнавания
в терминах Байесовой теории, которая была предложена Теодором В. Андерсоном,
и предоставить методы, которые позволят реализовать решение.
Это даст возможность использовать корректно распознанные модели
в системах биометнической идентификации.

Для достижения цели были использованы
\begin{enumerate}
  \item
    Базелевская модель головы была использована
    как порождающая модель поверхности лица.
  \item
    Байесовая теория распознавания образов была использована
    для правильной постановки и решения задачи и анализа существующих решений;
  \item
    Численные методы были использованы для расчёта решения.
\end{enumerate}

\MakeUppercase{Байесовая стратегия, распознавание образов,
пространственная реконструкция, порождающая модель,
биометрическая идентификация}
