\section{Задача}

\subsection{Початкові умови}

\subsubsection{Зображення}

Позначимо множину $T$ зображень і множину $C$ кольорів.
Введемо множину $I$ пікселів зображення.
Зображення $t \in T$ є відображення з множини пікселів на множину їх значень
\begin{equation*}
  t: I \rightarrow C.
\end{equation*}
Колір пікселя $i$ в зображенні $t$ позначимо як $t_i$.

Взагалі кажучи, $I$ --- множина індексів матриць однакового розміру
\begin{equation*}
  I = \left\{ \left\langle i, j \right\rangle
    \;\middle|\; i = 1..h,\; j = 1..w \right\}.
\end{equation*}
Зазвичай використовуються зображення розміром від
$100 \times 100 = 10^4$ пікселів.
Проте сучасні камери на фотоапаратах, смартфонах та інших пристроях
можуть зробити зображення площею кілька мільйонів пікселів і більше.
При використанні $2^8 = 256$ градацій сірого маємо
$2^{8 \cdot 10^4} \approx 10^{24 \cdot 10^3}$
різних зображень розміром $100 \times 100$, тобто неймовірно багато.

Тривимірна модель обличчя визначається не тільки набором $n$ дійсних параметрів
з множини $X = \mathbb{R}^n$.
Є додаткові параметри, що відіграють важливу роль при візуалізації,
проте не потрібні в результаті --- поворот, масштаб, перспектива тощо.
Введемо відображення $\theta^M \in \Theta^M$,
що застосовує необхідні перетворення до точки в тривимірному просторі
\begin{equation*}
  \theta^M: \mathbb{R}^3 \rightarrow \mathbb{R}^3.
\end{equation*}
Функцією, що перетворює набір параметрів на зображення, є відображення
\begin{equation*}
  f: \Theta^M \times X \rightarrow T.
\end{equation*}
Зображення $t$, згенероване з параметрами $\theta^M$ та $x$, позначатимемо
\begin{equation*}
  f_{\theta^M} \left( x \right) = t.
\end{equation*}
В подальших записах індекс $\theta^M$ записувати не будемо,
коли з контексту буде зрозуміло, які перетворення застосовуються до моделі.

Для подальших обчислень введемо функцію $h$ (hidden),
яка приймає значення $0$,
якщо в даній точці зображення є піксель обличчя,
та дорівнює $1$ у протилежному випадку
\begin{align*}
  h: \Theta^M \times X \rightarrow \left\{ 0, 1 \right\}.
\end{align*}
Введемо короткий запис для інверсії маски
\begin{align*}
  \overline{h}^{\theta^M}_i\left( x \right)
  = 1 - h^{\theta^M}_i\left( x \right),\qquad
  i \in I.
\end{align*}

\subsubsection{Опорні точки}

Оскільки відповідні вершини всіх моделей мають однакове семантичне значення,
за рахунок чого простір згенерованих облич і є опуклим,
можна скористатися інформацією деяких особливих з точки зору людини точок.
Такі точки як краї губ, носа, очей і таке інше назвемо опорними точками.
Множину обраних точок в контексті певної задачі позначатимемо
\begin{equation*}
  L \subset V.
\end{equation*}
Положення $u$ опорної точки $l \in L$ у моделі $m$ позначимо
\begin{equation*}
  m_{l} = u.
\end{equation*}
Відображення тривимірної точки,
до якої застосовано перетворення $\theta^M$,
\begin{equation*}
  P_2: \Theta^M \times \mathbb{R}^3 \rightarrow I.
\end{equation*}
Проекцію $p$ точки $v$ моделі $m$ на площину запишемо
\begin{equation*}
  P^2_{\theta^M}\left( m_v \right) = p.
\end{equation*}
Також, коли з контексту зрозуміло значення $\theta^M$,
позначатимемо без нижнього індексу
\begin{equation*}
  P^2\left( m_v \right) = p.
\end{equation*}
Введемо відображення $\theta^L$,
що для певної опорної точки $l$
дає координати цієї точки на даному зображенні $t$
\begin{equation*}
  \theta^L: L \times T \rightarrow I.
\end{equation*}
Координати опорної точки $l$ на зображенні $t$ позначимо
\begin{equation*}
  \theta_l^L\left( t \right) = i.
\end{equation*}
Аргумент використовувати не будемо, коли з контексту буде зрозуміло,
про яке зображення йде мова.

Потрібна метрика,
що буде відображати віддаленість двох точок на множині пікселів зображення.
Той факт,
що дві точки $i$ і $i'$ знаходяться одна від одної на відстані $\mu$,
запишемо як
\begin{equation*}
  \left\| i - i' \right\| = \mu.
\end{equation*}
Оскільки $i$ --- впорядкована пара чисел,
що відповідають координатам певного пікселя,
природнім чином можно впровадити звичайну евклідову метрику
для розрахунку відстані між двома точками зображення.

\subsubsection{Фон}
Якщо зображення складається не лише з пікселів обличчя,
обов'язково є фон.
Він може бути відомим, наприклад,
якщо заздалегіть було отримано зображення фону,
на якому буде відбуватися фотозйомка.
У загальному випадку він є невідомою величиною,
розподіл якої також невідомий,
а може не існувати зовсім.
Позначимо його як зображення $\theta^B$
\begin{align*}
  \theta^B \in T.
\end{align*}
Фон має такий самий розмір, як і вхідне зображення.

\subsubsection{Шум}

Вважаємо,
що на вхідне зображення накладено шум $\eta$,
що є вектором незалежних випадкових величин,
розподілених за центрованим нормальним законом
з однаковою невідомою дисперсією $\sigma_t^2$.
Колір пікселя $i$
\begin{equation*}
  t_i = f_i\left( x \right) \cdot \overline{h}_i\left( x \right)
    + \theta_i^B \cdot h_i\left( x \right) + \eta_i,\qquad
  \eta_i \sim \mathcal{N}\left( 0, \sigma^2_t\right), \qquad
  i \in I.
\end{equation*}

Замість реального положення опорних точок $\theta^L$
будемо розглядати її оцінку $\hat{\theta}^L$,
яка отримана з похибкою,
що має центрований гаусовий розподіл
з дисперсією $\sigma^2_L$
\begin{equation*}
  \hat{\theta}^L = \theta_t^L\left( l \right) + \zeta_l,
  \qquad \zeta_l \sim \mathcal{N}\left( 0, \sigma_L^2 \right).
\end{equation*}

Позначимо відхилення проекції опорних точок моделі
від їх оцінки
\begin{align*}
  \Delta\left( x \right)
  = P^2_{\theta^M}\left( G_L\left( x \right) \right) - \hat{\theta}^L.
\end{align*}

Ймовірність того,
що буде пред'явлено зображення $t$ з параметрами $x$
при відомих перетвореннях $\theta^M$
і оцінці положеннь опорних точок $\hat{\theta}^L$, є сумісною ймовірністю
\begin{equation*}
  p_{\theta}
  = \mathbb{P}_{\theta}\left\{
    f_{\theta^M}\left( \xi \right) + \eta = t,
    \zeta = \Delta\left( \xi \right),
    \xi = x
  \right\}.
\end{equation*}
З визначення умовної ймовірності випливає
\begin{equation*}
  p_{\theta}
  = \mathbb{P}_{\theta}\left\{
      f_{\theta^M}\left( \xi \right) + \eta = t,
      \zeta = \Delta\left( \xi \right)
      \;\middle|\; \xi = x \right\}
    \cdot \mathbb{P}\left( \xi = x \right)
\end{equation*}
Позбавимося умовної ймовірності, замінивши $\xi$ на $x$.
Також розіб'ємо перший множник на дві ймовірності,
бо ці події не залежать одна від одної
\begin{equation*}
  p_{\theta}
  = \mathbb{P}_{\theta^M}\left( \eta = t - f_{\theta^M}\left( x \right) \right)
    \cdot \mathbb{P}_{\theta}\left\{
      \zeta = \Delta\left( x \right)
    \right\}
    \cdot \mathbb{P}\left( \xi = x \right).
\end{equation*}

Далі будемо позначати сумісну ймовірність без $\theta$,
коли з контексту буде зрозуміло,
які параметри використані
\begin{align*}
  \mathbb{P}\left( t, x \right) = p_{\theta}.
\end{align*}

\subsection{Узагальнення на випадок кількох зображень}

Пред'явлено кілька зображень
\begin{equation*}
  \vec{t} = \left\langle t_1, t_2, \dots, t_m \right\rangle,
\end{equation*}
на яких знаходиться обличчя однієї й тієї ж самої людини.
Для спрощення вважаємо,
що форма обличчя (і колір шкіри, якщо потрібно)
на всіх зображеннях однаковий, тобто $x = const \in X$.
Цього важко досягти,
якщо обрати фотографії зняті з різних камер в різні пори року,
проте ця вимога має місце,
коли на вході дано відео, фотографії з однієї фотосесії,
фотографії зроблені в один момент з різних ракурсів,
або розрізане зображення, де людину видно у рівних чистих дзеркалах.
Для різних зображень $t^j$ буде відрізнятися
шум $\eta^j$ та параметри зйомки $\theta^j$.

Набір зображень генерується стохастичним автоматом \cite{Rabin:1963}
\begin{equation*}
  \mathfrak{U}_x = \left\langle
    \Theta \times T \times X
      \cup \left\{ \varepsilon_0, \varepsilon \right\}, p, \varepsilon_0,
    \left\{ \varepsilon \right\}
  \right\rangle,
\end{equation*}
де $\varepsilon_0$ --- початковий стан, який не містить ніякої інформації,
а $\varepsilon$ --- кінцевий стан.
Функція $p$ визначає ймовірність того,
що наступним набором параметрів буде $\theta_j$
та згенерується зображення $t_j$.
Якщо зображення не залежать одне від одного
(рис. \ref{fig:solutions:tree-images}),
маємо ймовірності станів,
що не залежать один від одного
\begin{equation*}
  \mathbb{P}_j\left( \theta^j, t^j, x \right)
  = p\left( \theta^j, t^j \right).
\end{equation*}
У випадку відео (рис. \ref{fig:solutions:tree-video})
ймовірноість генерації наступних параметрів та зображення
залежить від поточного стану автомата
\begin{equation*}
  \mathbb{P}_j\left( \theta^j, t^j, x \right)
  = p_j\left( \theta^j, t^j \; \mid \; \theta^{j-1} \right).
\end{equation*}

\begin{figure}[h]
  \centering
  \includestandalone[mode=buildnew]{../tikz/5_40_images_tree}
  \caption{Приклад роботи автомата з незалежними станами}
  \label{fig:solutions:tree-images}
\end{figure}

\begin{figure}[h]
  \centering
  \includestandalone[mode=buildnew]{../tikz/5_40_video_tree}
  \caption{Приклад автомата, що генерує відео}
  \label{fig:solutions:tree-video}
\end{figure}

\subsection{Баєсова задача розпізнавання}

Поставимо баєсову задачу розпізнавання \cite{Anderson:1963}.
Для цього потрібно визначитися з функцією витрат
\cite{berger1980}
\begin{equation*}
  W: X \times X \rightarrow \mathbb{R}.
\end{equation*}
Стратегію
\begin{equation*}
  q: T \rightarrow X,
\end{equation*}
яка для зображення $t$ дає результат $x$, позначимо
\begin{equation*}
  q\left( t \right) = x.
\end{equation*}
Математичне очікування функції витрат $W$
як функції випадкової пари $\left\langle t, x \right\rangle$
для даного вирішального правила $q$ називається баєсовим ризиком
\cite{wald1955selected}
\begin{equation*}
  R_{\theta} \left( q \right)
  = \int\limits_{T, X}
    W \left( x, q\left( t \right) \right)
    d F_{\theta}\left( t, x \right)
  = \Meanof{\theta}{W\left( x, q\left( t \right) \right)},
\end{equation*}
де $F_{\theta}$ --- функція сумісного розподілу зображення
та параметрів зображеної моделі при відомих $\theta$.

\subsection{Баєсова стратегія}

Стратегія $q^*$ називається баєсовою,
якщо може бути представлена у вигляді \cite{schlezinger:2013}
\begin{equation}\label{eq:bayesian}
  q^*
  = \argmin\limits_{q \in Q}
    \sum_{\theta \in \Theta}
      \tau\left( \theta \right) \cdot R_{\theta}\left( q \right),
  \qquad
  \tau\left( \theta \right) \ge 0,
  \qquad
  \sum_{\theta \in \Theta} \tau\left( \theta \right) = 1.
\end{equation}
Стратегії іншого вигляду є негодящими:
якщо $q_0$ --- стратегія,
яку не можна представити у вигляді \eqref{eq:bayesian},
обов'язково знайдеться баєсова стратегія $q$ така, що
при будь-якому значенні параметрів $\theta$
небаєсова стратегія буде розпізнавати гірше за певну баєсову
\begin{equation*}
  R_{\theta}\left( q_0 \right) > R_{\theta}\left( q \right), \qquad
  \forall \theta \in \Theta.
\end{equation*}

Оскільки зовнішній вигляд найкращих стратегій декларовано,
для кожної задачі треба шукати лише коефіцієнти $\tau\left( \theta \right)$.
Прикладом доцільної задачі у даному випадку
може бути пошук такої стратегії,
що помиляється якнайменше при кожному $\theta$.
За визначенням мінімальні втрати гарантує та стратегія,
якій заздалегіть відомо параметр $\theta$, з яким вона працюватиме.
Така стратегія називається оптимальною
\begin{equation*}
  q^{\theta}\left( t \right)
  = \argmin\limits_{q \in Q} R_{\theta}\left( q \right).
\end{equation*}

Далі зафіксуємо параметр $\theta$, ніби він відомий.

\subsection{Бінарна функція витрат}

Досить розповсюдженою, проте зазвичай неприродною є бінарна штрафна функція
\begin{equation*}
  W \left( x, x' \right)
  = \mathbbm{1} \left( x \neq x' \right).
\end{equation*}
Для неперервного випадку така функція витрат не підходить,
бо баєсів ризик
\begin{equation*}
  R_{\theta}\left( q \right)
  = \sum\limits_{t \in T}
    \int\limits_{x \in X}
    \mathbbm{1} \left( x \neq q\left( t \right) \right)
    \cdot p\left( t, x \right) d\; x
  = 1,\qquad
  \forall q \in Q.
\end{equation*}
Будь-яка стратегія дає невірну відповідь у неперервному випадку
з точки зору бінарної функції витрат,
тому для неї буде розглядатися лише дискретний випадок.

Оберемо стратегію $q^*$,
що мінімізує математичне очікування цієї функції витрат
\begin{equation*}
  \begin{split}
    q^*_{\theta}\left( t \right)
    = \argmin_{x'} \left\{
      \sum\limits_{x \in X}
        \mathbb{P}\left( t, x \right)
        \cdot \mathbbm{1} \left( x \neq x' \right)
      \right\} = \\
    = \argmin_{x'} \left\{
      \sum\limits_{x \in X}
        \mathbb{P}\left( t, x \right)
      - \sum\limits_{x \in X}
        \mathbb{P}\left( t, x \right)
        \cdot \mathbbm{1} \left( x = x' \right)
      \right\} = \\
    = \argmin_{x'} \left\{
      1 - \mathbb{P}\left( t, x' \right)
      \right\}.
  \end{split}
\end{equation*}
В результаті
\begin{equation*}
  q^*_{\theta}\left( t \right)
  = \argmax_x \mathbb{P} \left( t, x \right).
\end{equation*}

Отже, якщо використовується бінарна функція витрат,
потрібно обирати найбільш ймовірний набір параметрів.
Аналітичний вираз для розрахування $f$ досить складний,
тому доведеться скористатися чисельними методами,
які не завжди дають точну відповідь
і можуть зупинитися у точці локального отптимуму, якщо такий є
і не співпадає з глобальним.

\subsection{Інтервальна функція витрат}

Більш загальним варіантом бінарної штрафної функції є
індикатор неналежності дійсного параметру $x$
деякій $\delta$-околиці обраного параметру $x'$
\begin{equation*}
  W \left( x, x' \right)
  = \mathbbm{1} \left( x \notin \delta\left( x' \right) \right).
\end{equation*}

Оберемо стратегію, що мінімізує математичне очікування цієї функції витрат
\begin{equation*}
  \begin{split}
    q^*_{\theta}\left( t \right)
    = \argmin_{x'} \left\{
      \int\limits_{X}
        \mathbbm{1} \left( x \notin \delta\left( x' \right) \right)
        d F\left( t, x \right)
      \right\} = \\
    = \argmin_{x'} \left\{
      1 -
      \int\limits_{X}
        \mathbbm{1} \left( x \in \delta\left( x' \right) \right)
        d F\left( t, x \right)
      \right\}
  \end{split}
\end{equation*}
В результаті
\begin{equation*}
  q^*_{\theta}\left( t \right)
  = \argmax_{x'} \int\limits_{\delta\left( x' \right)}
    dF_{\theta}\left( t, x \right)
  = \argmax_{x'} \mathbb{P} \left( t, x \in \delta\left( x' \right) \right).
\end{equation*}
Потрібно обрати таку точку,
щоб ймовірність того,
що параметри належать її $\delta$-околиці,
була більшою,
ніж для $\delta$-околиць інших точок при даному зображенні $t$.

\subsection{Різниця моделей}

Розглянемо більш природню функцію витрат ---
квадрат евклідової відстані між точками дійсної та обраної моделі.

Функція витрат має вигляд
\begin{equation*}
  \begin{split}
    W \left( x, x' \right)
    = \left\| G\left( x \right) - G\left( x' \right) \right\|^2
    = \sum_{v \in V} \left[
        G_v\left( x \right) - G_v\left( x' \right)
      \right]^2 = \\
    = \sum_{v \in V} \sum_{i = 1}^n \left[
        \lambda_i^v \cdot \left( x_i - x'_i \right)
      \right]^2
    = \sum_{i = 1}^n \left\{ \left( x_i - x'_i \right)^2
      \cdot \sum_{v \in V} \left( \lambda_i^v \right)^2 \right\} = \\
    = \left| \beta_i^2 = \sum_{v \in V} \left( \lambda_i^v \right)^2 \right|
    = \sum_{i = 1}^n \beta_i^2 \cdot \left( x_i - x'_i \right)^2.
  \end{split}
\end{equation*}

Оберемо стратегію $q^*$,
що мінімізує математичне очікування цієї функції витрат
\begin{equation*}
  q^*_{\theta}\left( t \right)
  = \argmin_{x'} \left\{
    \int\limits_{X}
        \sum_{i = 1}^n \beta_i^2 \cdot \left( x'_i - x_i \right)^2
        dF_{\theta}\left( t, x \right)
    \right\}.
\end{equation*}
Щоб мінімізувати неперервну функцію від параметрів $x'_i$,
можна взяти по них похідну
(ми можемо це зробити, бо $x'_i$ не є змінною,
по якій ведеться інтегрування --- з точки зору інтегралу це лише константа)
\begin{equation*}
  \frac{\partial \int\limits_{X}
      \sum\limits_{i = 1}^n \beta_i^2 \cdot \left( x'_i - x_i \right)^2
      dF_{\theta}\left( t, x \right)
  }{\partial x'_i}
  = 2 \cdot \int\limits_{X}
    \beta_i^2 \cdot \left( x'_i - x_i \right) dF_{\theta}\left( t, x \right),
    \qquad i = 1..n
\end{equation*}
та прирівняти до нуля
\begin{equation*}
  \int\limits_{X} \left( x'_i - x_i \right) dF_{\theta}\left( t, x \right) = 0,
  \qquad i = 1..n.
\end{equation*}
Значення компоненти
\begin{equation*}
  x'_i
  = \frac{\int\limits_{X} x_i dF_{\theta}\left( t, x \right)}
         {\int\limits_{X} dF_{\theta}\left( t, x \right)}
  = \frac{\int\limits_{X} x_i dF_{\theta}\left( t, x \right)}
         {\mathbb{P}\left( t \right)}
\end{equation*}
Результуюча стратегія
\begin{equation*}
  q^*_{\theta}\left( t \right)
  = \frac{\int\limits_{X} x dF_{\theta}\left( t, x \right)}{\mathbb{P}\left( t \right)}.
\end{equation*}
У вигляді умовного математичного очікування
\begin{equation*}
  q^*_{\theta}\left( t \right) = \Meanof{\theta}{x \mcond t}.
\end{equation*}

Ділення на ймовірність появи зображення $t$ можна сприймати як нормування
при використанні чисельних методів розрахунку даного інтегралу.

\subsection{Різниця параметрів}

Розглянемо більш просту функцію витрат ---
квадрат евклідової норми різниці між дійсними та обраними параметрами
моделі зображеного обличчя
\begin{equation*}
  W \left( x, x' \right)
  = \left\| x - x' \right\|^2
  = \sum_{i = 1}^n \left( x_i - x'_i \right)^2.
\end{equation*}

Оберемо стратегію $q^*$,
що мінімізує математичне очікування цієї функції витрат
\begin{equation*}
  q^*_{\theta}\left( t \right)
  = \argmin_{x'} \left\{
    \int\limits_{X}
        \sum_{i = 1}^n \left( x'_i - x_i \right)^2
        dF_{\theta}\left( t, x \right)
    \right\}.
\end{equation*}
Маємо мінімізацію неперервної функції від параметрів $x'_i$,
отже можемо взяти по них похідну
\begin{equation*}
  \frac{\partial \int\limits_{X}
      \sum\limits_{i = 1}^n \left( x'_i - x_i \right)^2
      dF_{\theta}\left( t, x \right)
  }{\partial x'_i}
  = 2 \cdot \int\limits_{x \in X} \left( x'_i - x_i \right)
    dF_{\theta}\left( t, x \right), \qquad i = 1..n
\end{equation*}
та прирівняти до нуля
\begin{equation*}
  \int\limits_{X}
    \left( x'_i - x_i \right) dF_{\theta}\left( t, x \right) = 0, \qquad i = 1..n.
\end{equation*}
Значення компоненти
\begin{equation*}
  x'_i
  = \frac{\int\limits_{X} x_i dF_{\theta}\left( t, x \right)}
         {\int\limits_{X} dF_{\theta}\left( t, x \right)}
  = \frac{\int\limits_{X} x_i dF_{\theta}\left( t, x \right)}
         {\mathbb{P}\left( t \right)}
\end{equation*}
Результуюча стратегія
\begin{equation*}
  q^*_{\theta}\left( t \right)
  = \frac{\int\limits_{X} x dF_{\theta}\left( t, x \right)}{\mathbb{P}\left( t \right)}.
\end{equation*}
Оптимальна стратегія як умовне математичне очікування
\begin{equation*}
  q^*_{\theta}\left( t \right) = \Meanof{\theta}{x \mcond t}.
\end{equation*}

Отримана та ж стратегія,
що мінімізує математичне очікування суми квадратів різниць
координат вершин дійсної та обраної моделі обличчя.
Тобто це вирішувальне правило розв'язує обидві задачі.

Ділення на ймовірність появи зображення $t$ можна сприймати як нормування
при використанні чисельних методів розрахунку даного інтегралу,
як і в минулому прикладі.
