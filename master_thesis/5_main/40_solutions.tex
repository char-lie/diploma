\section{Розв'язок}

\subsection{Дискретизація простору параметрів}

Дискретизуємо простір параметрів $X$.
Введемо невелике $\delta x$ таке,
щоб ймовірність того, що значення потрапляє в рамки гіперкубу
$\left[ x - \frac{\delta x}{2}; x + \frac{\delta x}{2} \right]$,
приблизно дорівнювала добутку щільності розподілу в його центрі
на його об'єм
\begin{equation*}
  \mathbb{P}\left( \xi \in \left[ x - \frac{\delta x}{2};
                                  x + \frac{\delta x}{2} \right] \right)
  \approx p\left( x \right) \cdot \delta x^n,
\end{equation*}
де
\begin{equation*}
  \left( x \pm \frac{\delta x}{2} \right)_i = x_i \pm \frac{\delta x}{2},\qquad
  i = 1..n.
\end{equation*}
Далі треба розбити простір на гіперкуби
зі стороною $\delta x$ і вважати,
що коли величина потрапляє на територію певного гіперкубу,
то вона дорівнює значенню в його центрі.

Відомо, що параметри $x$ --- набори коефіцієнтів нормованих головних компонент,
тобто мають стандартний гаусовий розподіл.
Ймовірність того, що було згенеровано саме такий набір
\begin{equation*}
  \mathbb{P}_X\left( x \right)
  = \delta x^n \cdot \prod_{i=1}^n
    \frac{\exp{\left( - \frac{x_i^2}{2} \right)}}{\sqrt{2 \cdot \pi}}.
\end{equation*}

Введемо зручне позначення для шуму в пікселі $i$
\begin{equation*}
  y_i = t_i - f_i\left( x \right).
\end{equation*}
Розглядається одне зображення $t$.
З контексту буде зрозуміло, яке значення має $x$.
Ймовірність того, що обличчя на даному зображенні отримано з параметрами $x$
\begin{equation*}
  \mathbb{P}_Y\left( t \;\middle|\; x \right)
  = \delta x^{w \cdot h} \cdot \prod_{i \in I}
    \frac{\exp{\left\{- \frac{ y_i^2}
           {2 \cdot \sigma^2_t} \right\}}}
           {\sqrt{2 \cdot \pi \cdot \sigma^2_t}}.
\end{equation*}

Відхилення проекції опорної точки $l$ моделі,
отриманої з параметрами $x$,
від реального положення опорної точки на зображенні,
позначимо $\Delta_l$
\begin{equation*}
  \Delta_l\left( x \right)
  = P^2_{\theta_M}\left( G_l\left( x \right) \right) - \theta_l,
  \qquad l \in L.
\end{equation*}
Ймовірність того, що параметри $x$ приймають саме такі значення,
якщо відомі координати опорних точок,
\begin{equation*}
  \mathbb{P}_L\left( x \right)
  = \delta x^{\left| L \right|} \cdot \prod_{l \in L}
      \frac{\exp{\left\{- \frac{\Delta_l^2\left( x \right)}
           {2 \cdot \sigma^2_L} \right\}}}
           {\sqrt{2 \cdot \pi \cdot \sigma^2_L}}.
\end{equation*}

Сумісна ймовірність зображення $t$ та параметрів $x$
\begin{equation*}
  \mathbb{P}\left( x, t \right)
  = \mathbb{P}_Y\left( t \;\middle|\; x \right)
    \cdot \mathbb{P}_X\left( x \right)
    \cdot \mathbb{P}_L\left( x \right)
\end{equation*}

\subsection{Інтервальна функція витрат}

Розглянемо функцію витрат,
за якої вірним є будь-який набір параметрів,
що знаходиться в $\delta$-околі дійсного набору $x$,
а за всі інші сплачується штраф $1$.

Для максимізації розглянемо логарифми ймовірностей,
тому що це зручніше,
ніж максимізація добутку кількох десятків тисяч або мільйонів значень
\begin{equation*}
  \begin{split}
    \mathbb{P}_X\left( x \right)
    &= \ln{\delta x^n} +
        \sum_{i = 1}^n
        \left\{
          - \frac{x_i^2}{2}
          - \frac{\ln{2} + \ln{\pi}}{2}
        \right\}, \\
    \mathbb{P}_Y\left( t \;\middle|\; x \right)
    &= \ln{\delta x^{w \cdot h}} +
        \sum_{i \in I}
        \left\{
          - \frac{y_i^2}{2 \cdot \sigma^2}
          - \frac{\ln{2} + \ln{\pi} + \ln{\sigma^2}}{2}
        \right\}, \\
    \mathbb{P}_L\left( x \right)
    &= \ln{\delta x^{\left| L \right|}} +
        \sum_{l \in L}
        \left\{
          - \frac{\Delta_l^2\left( x \right)}{2 \cdot \sigma_L^2}
          - \frac{\ln{2} + \ln{\pi} + \ln{\sigma_L^2}}{2}
        \right\}.
  \end{split}
\end{equation*}
Приберемо константні доданки,
що не впливають на результат максимізації ймовірності
появи даного зображення $t$ з даними параметрами $x$
\begin{equation*}
  \sum_{i \in I}
    \left\{
      - \frac{y_i^2}{2 \cdot \sigma^2}
      - \frac{\ln{\sigma^2}}{2}
    \right\}
  +
  \sum_{l \in L}
    \left\{
      - \frac{\Delta_l^2\left( x \right)}{2 \cdot \sigma_L^2}
      - \frac{\ln{\sigma_L^2}}{2}
    \right\}
  - \sum_{i = 1}^n \frac{x_i^2}{2}
  \to \max\limits_{x \in X}.
\end{equation*}
Помножимо на $-2$ та винесемо логарифм дисперсії за знак суми
\begin{equation}\label{eq:minimize}
  \left| I \right| \cdot \ln{\sigma^2}
  + \left| L \right| \cdot \ln{\sigma_L^2}
  + \sum_{i \in I} \frac{ y_i^2}{\sigma^2}
  + \sum_{l \in L} \frac{ \Delta_l^2\left( x \right)}{\sigma_L^2}
  + \sum_{i = 1}^n x_i^2
  \to \min\limits_{x \in X}.
\end{equation}

Відомо, що $n$ набагато менше за $w \cdot h$.
Визначимо, за яких умов можна знехтувати ймовірністю набору параметрів $x$,
якщо не враховуються опорні точки.
Для деякого $\varepsilon$ повинна виконуватись нерівність
\begin{equation*}
  \left|
    \frac{\sum\limits_{i = 1}^n x_i^2}
         {w \cdot h \cdot \ln{\sigma^2}
          + \sum\limits_{i \in I}
            \frac{ y_i^2}{\sigma^2}}
  \right|
  < \varepsilon.
\end{equation*}
Розв'яжемо цю нерівність відносно дисперсії шуму
\begin{equation*}
  \frac{\sum\limits_{i = 1}^n x_i^2}{\varepsilon \cdot w \cdot h}
  < \left| \ln\sigma^2
    + \frac{\sum\limits_{i \in I} y_i^2}
           {\sigma^2 \cdot w \cdot h}
   \right|.
\end{equation*}
Права частина містить суму, яка дорівнює оцінці дисперсії
діленій на реальну дисперсію шуму.
Вважаємо, що зображення має достатньо пікселів,
щоб сума була приблизно рівною одиниці
\begin{equation*}
  \frac{\sum\limits_{i = 1}^n x_i^2}{\varepsilon \cdot w \cdot h}
  < \left| \ln\sigma^2 + 1 \right|.
\end{equation*}
Потрібно позбутися модуля.
Якщо $\ln\sigma^2 + 1 > 0$, то
\begin{equation*}
  \sigma^2 > e^{-1} \approx 0.37.
\end{equation*}
Це завеликий шум, який сильно спотворить зображення,
тому не будемо розглядати цей випадок.
Отримали нерівність
\begin{equation*}
  - \frac{\sum\limits_{i = 1}^n x_i^2}{\varepsilon \cdot w \cdot h} - 1
  > \ln\sigma^2.
\end{equation*}
Візьмемо експоненту від обох частин
\begin{equation*}
  \sigma^2
  < \exp{\left\{
      - \frac{\sum\limits_{i = 1}^n x_i^2}
             {\varepsilon \cdot w \cdot h}
      - 1
    \right\}}.
\end{equation*}
Перепишемо суму квадратів параметрів моделі як їх середнє помножене на кількість
\begin{equation*}
  \sigma^2
  < \exp{\left\{
      - \frac{\left\| x \right\|^2}
             {\varepsilon \cdot w \cdot h}
      - 1
    \right\}}.
\end{equation*}
Якщо в нерівність підставити максимально можливе значення для
$\left\| x \right\|^2$,
отримаємо більш жорсткі обмеження на дисперсію шуму.
Такий прийом лише збільшить точність результату.
\begin{equation*}
  \sigma^2
  < \exp{\left\{
      - \frac{n \cdot x_{max}^2}
             {\varepsilon \cdot w \cdot h}
      - 1
    \right\}}.
\end{equation*}
Нехай $\varepsilon$ лінійно залежить від максимально можливого значення
параметрів $x$.
Тоді
\begin{equation*}
  \sigma^2
  < \exp{\left\{
      - \frac{n}{\varepsilon' \cdot w \cdot h}
      - 1
    \right\}},
    \qquad \varepsilon = \varepsilon' \cdot x_{max}^2.
\end{equation*}
Параметри $x$ мають стандартний нормальний розподіл,
отже за правилом трьох сігма не варто брати значення,
що за модулем перевищують 3.
Поріг $\varepsilon$ достатньо взяти $0.1$ ---
це буде означати, що доданком,
який вдесятеро менший за інший доданок,
можна знехтувати.
Тоді $\varepsilon' \approx 0.01$.

Якщо вважати шум з дисперсією $e^{-1.2} \approx 0.3$ завеликим
(таким, що не зустрінеться на реальних зображеннях),
отримаємо залежність максимальної кількості параметрів
від розміру зображення
\begin{equation}\label{eq:negligible_x}
  n
  < 0.2 \cdot \varepsilon' \cdot w \cdot h
  \le \frac{w \cdot h}{500}
  \qquad \varepsilon' \le 0.01.
\end{equation}
Тоді буде виконуватись умова
\begin{equation*}
  \sigma^2 < e^{-1.2}.
\end{equation*}

Повернемося до \eqref{eq:minimize}та замінимо дисперсію її оцінкою
\cite{BorovkovMS}
\begin{equation*}
  \left| I \right|
  \cdot \ln{
    \frac{\sum\limits_{i \in I} y_i^2}
         {\left| I \right| - 1}}
  + \left| I \right| - 1
  + \left| L \right|
  \cdot \ln{
    \frac{\sum\limits_{l \in L} \Delta_l^2\left( x \right)}
         {\left| L \right| - 1}}
  + \left| L \right| - 1
  + \sum_{i = 1}^n x_i^2
  \to \min\limits_{x \in X}.
\end{equation*}
Позбавимося константних доданків
\begin{equation*}
  \left| I \right|
  \cdot \ln{
    \frac{\sum\limits_{i \in I} y_i^2}
         {\left| I \right| - 1}}
  + \left| L \right|
  \cdot \ln{
    \frac{\sum\limits_{l \in L} \Delta_l^2\left( x \right)}
         {\left| L \right| - 1}}
  + \sum_{i = 1}^n x_i^2
  \to \min\limits_{x \in X}.
\end{equation*}
В загальному випадку зображення містить не тільки обличчя.
Генерація усіх можливих фонів для кожної моделі є недоцільною.
Потрібно отримати формулу,
яка не прив'язана до кожного пікселя вхідного зображення $t$.
Для цього перепишемо доданки як середньоквадратичні відхилення
\begin{equation*}
  \left| I \right|
  \cdot \ln{\widetilde{y}^2}
  + \left| L \right|
  \cdot \ln{\left[ \widetilde{\Delta}^2\left( x \right) \right]}
  + \left( n - 1 \right) \cdot \widetilde{x}^2
  \to \min\limits_{x \in X}.
\end{equation*}
Треба брати середньоквадратичне відхилення шуму по тим пікселям,
де на згенерованому зображенні $f\left( x \right)$ знаходиться обличчя.
За умовою накладений шум є
набором незалежних випадкових однаково розподілених величин,
тому квадрат середньоквадратичного відхилення будь-якої їх сукупності
буде оптимальною оцінкою їх дисперсії.
Це означає, що квадрат середньоквадратичного відхилення шуму
на обличчі буде тим точнішою оцінкою дисперсії шуму,
чим більше місця займає обличчя на зображенні.

Знехтуємо ймовірністю параметрів $x$
\begin{equation*}
  \widetilde{y}^2 \to \min\limits_{x \in X}.
\end{equation*}
Помножимо на $\left( w \cdot h - 1 \right)$.
Отримали мінімізацію відомої цільової функції --- суми квадратів відхилень
\begin{equation*}
  \sum_{i \in I} y_i^2
  = \sum_{i \in I} \left( t_i - f_i\left( x \right) \right)^2
  \to \min\limits_{x \in X}.
\end{equation*}
Мінімізація суми квадратів різниць між дійсним та згенерованим зображенням
розв'язує задачу з бінарною або інтервальною функцією штрафу
та гаусовим шумом на зображенні без додаткових умов
на кшталт розподілу параметрів $x$.
Проте це знехтування доцільне при виконанні умов \eqref{eq:negligible_x}.

Продивимось знову отримані формули і їх особливості.

Без додаткових припущень маємо
\begin{equation*}
  \left| I \right| \cdot \ln{\sigma^2}
  + \left| L \right| \cdot \ln{\sigma_L^2}
  + \sum_{i \in I} \frac{ y_i^2}{\sigma^2}
  + \sum_{l \in L} \frac{ \Delta_l^2\left( x \right)}{\sigma_L^2}
  + \sum_{i = 1}^n x_i^2
  \to \min\limits_{x \in X}.
\end{equation*}

Оскільки дисперсія є константним невідомим параметром,
можна спростити
\begin{equation*}
  \sum_{i \in I} \frac{ y_i^2}{\sigma^2}
  + \sum_{l \in L} \frac{ \Delta_l^2\left( x \right)}{\sigma_L^2}
  + \sum_{i = 1}^n x_i^2
  \to \min\limits_{x \in X}.
\end{equation*}

Якщо замість сум квадратів розглядати вибіркові дисперсії,
щоб позбутися невідомих дисперсій, задача приймає вигляд
\begin{equation*}
  \left| I \right|
  \cdot \ln{\widetilde{y}^2}
  + \left| L \right|
  \cdot \ln{\left[ \widetilde{\Delta}^2\left( x \right) \right]}
  + \left( n - 1 \right) \cdot \widetilde{x}^2
  \to \min\limits_{x \in X}.
\end{equation*}

Користуючись тим припущенням,
що оцінка дисперсії приблизно дорівнює реальній,
повернули логарифм дисперсії в формулу,
адже вона вже не константа, а замінена оцінкою, та отримали
\begin{equation*}
  w \cdot h \cdot \ln{\widetilde{y}^2}
  + \left( n - 1 \right) \cdot \widetilde{x}^2
  \to \min\limits_{x \in X}.
\end{equation*}

Найпростіша цільова функція --- сума квадратів відхилень
між дійсним та синтезованим зображенням.
Пригодна для випадку, відомо лише те, зображення має гаусовий шум.
\begin{equation*}
  \sum_{i \in I} y_i^2
  \to \min\limits_{x \in X}.
\end{equation*}


\subsection{Різниця параметрів}

Ймовірність того,
що дане зображення $t$ було отримано саме з параметрами $x$,
є добутком розглянутих ймовірностей
\begin{equation*}
  \mathbb{P}\left( x, t \right)
  = \mathbb{P}_I\left( t \mcond x \right)
    \cdot \mathbb{P}_X\left( x \right)
  = \delta x^{n + w \cdot h} \cdot \prod_{i \in I}
  \frac{\exp{\left\{ - \frac{\left( t_i - f_i\left( x \right) \right)^2}
       {2 \cdot \sigma_t^2} \right\}}}{\sqrt{2 \cdot \pi \cdot \sigma_t^2}}
  \cdot
  \prod_{i=1}^n
  \frac{\exp{\left\{ - \frac{x_i^2}{2} \right\}}}{\sqrt{2 \cdot \pi}}.
\end{equation*}
Винесемо константи з добутків
\begin{equation*}
  \mathbb{P}\left( x, t \right)
  = \left( \frac{\delta x}{\sqrt{2 \cdot \pi}} \right)^{n + w \cdot h}
    \cdot \prod_{i \in I}
    \frac{\exp{\left\{ - \frac{\left( t_i - f_i\left( x \right) \right)^2}
         {2 \cdot \sigma_t^2} \right\}}}{\sqrt{\sigma_t^2}}
    \cdot
    \prod_{i=1}^n
    \exp{\left\{ - \frac{x_i^2}{2} \right\}}.
\end{equation*}
В даному випадку не можна замінити дисперсію на її оцінку,
бо для кожного $x$ вона буде різною,
через що інтеграл ймовірностей не буде дорівнювати одиниці,
що порушує умову, якій повинна задовольняти ймовірнісна міра
\cite{dorogovtsev:1989}.
Введемо константу
\begin{equation*}
  c_t = \frac{
      \left( 2 \cdot \pi \right)^{- \frac{n + w \cdot h}{2}}
      \cdot \left( \sigma_t^2 \right)^{- \frac{w \cdot h}{2}}
    }{\mathbb{P}\left( t \right)}
\end{equation*}
Ймовірність певного набору параметрів $x$ на даному зображенні $t$
\begin{equation*}
  \mathbb{P}\left( x \mcond t \right)
  = c_t \cdot \delta x^{n + w \cdot h}
    \cdot \exp{\left\{ - \frac{\left\| t - f\left( x \right) \right\|^2}
                              {2 \cdot \sigma_t^2} \right\}}
    \cdot \exp{\left\{ - \frac{\left\| x \right\|^2}{2} \right\}}.
\end{equation*}
Стратегія розпізнавання
\begin{equation*}
  q^*_{\theta}\left( t \right)
  = c_t \cdot \delta x^{n + w \cdot h}
    \cdot \sum_{x \in X}
      x
      \cdot \exp{\left\{ - \frac{\left\| t - f\left( x \right) \right\|^2}
                                {2 \cdot \sigma_t^2} \right\}}
      \cdot \exp{\left\{ - \frac{\left\| x \right\|^2}{2} \right\}}.
\end{equation*}
Щільність розподілу набору параметрів $x$ на даному зображенні $t$
\begin{equation*}
  p\left( x \mcond t \right)
  = c_t
    \cdot \exp{\left\{ - \frac{\left\| t - f\left( x \right) \right\|^2}
                              {2 \cdot \sigma_t^2} \right\}}
    \cdot \exp{\left\{ - \frac{\left\| x \right\|^2}{2} \right\}}.
\end{equation*}
Стратегія розпізнавання у неперервному випадку
\begin{equation*}
  q^*_{\theta}\left( t \right)
  = c_t
    \cdot \int\limits_{x \in X}
      x
      \cdot \exp{\left\{ - \frac{\left\| t - f\left( x \right) \right\|^2}
                                {2 \cdot \sigma_t^2} \right\}}
      \cdot \exp{\left\{ - \frac{\left\| x \right\|^2}{2} \right\}}
    d\,x.
\end{equation*}
