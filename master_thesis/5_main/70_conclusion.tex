\chapterConclusion

Пред'явлено дві основні постановки, розв'язки яких суттєво відрізняються.
У випадку бінарної функції витрат на дискретизованій множині параметрів моделі
достатньо мінімізувати апостеріорну ймовірність шуканих величин.
Саме цей підхід і використовується у всіх відомих на даний момент методах.
Більш доречна для даного випадку функція витрат,
що дорівнює сумі квадратів відхилень отриманого рішення від реального,
потребує пошуку умовного математичного сподівання
шуканих величин за відомим фото.
Якщо вважати,
що глобальний мінімум цільової функції достатньо глибокий
і математичне сподівання буде приблизно дорівнювати екстремуму,
два дані підходи можна вважати близькими.

Той факт,
що для коректного розв'язку поставленої задачі треба шукати
математичне сподівання,
також вказує на те,
що знайдене рішення можна покращити за допомогою параметрів,
що знаходяться близько до розв'язку.

Одним з важливіших результатів є виявлення теоретичної некоректності підходу,
який полягає в тому,
що рішення шукається для найбільш ймовірних невідомих параметрів
(таких як освітлення, поворот тощо).
Це означає, що можна знайти стратегію,
яка при будь-яких значеннях невідомих параметрів буде давати менший ризик,
тобто буде менше помилятися.

Врахування фону призвело до альтернативної постановки задачі,
яка не була розглянута іншими дослідниками.

З минулих робіт виявлено,
як використовувати реконструкцію у системах біометричної ідентифікації.
