\section{Висновки}

Проведено аналіз задачі реконструкції та пред'явлено дві основні постановки,
розв'язки яких суттєво відрізняються.
У випадку бінарної функції витрат на дискретизованій множині параметрів моделі
достатньо мінімізувати апостеріорну ймовірність шуканих величин.
Саме цей підхід і використовується у всіх відомих на даний момент методах.
Більш адекватна для даного випадку функція витрат,
що дорівнює сумі квадратів відхилень отриманого рішення від реального,
потребує пошуку умовного математичного очікування
шуканих величин за відомим фото.
Якщо вважати, що глобальний мінімум цильової функції достатньо глибокий,
і математичне очікування буде приблизно дорівнювати екстремуму,
два дані підходи можна вважати близькими.

Той факт,
що для коректного розв'язку поставленої задачі треба шукати
математичне очікування,
також вказує на те,
що знайдене рішення можна покращити за допомогою врахування сусідніх параметрів.

Одним з важливіших результатів є виявлення теоретичної некоректності підходу,
який полягає в тому,
що рішення шукається для найбільш ймовірних невідомих параметрів
(таких як світлення, поворот тощо).
Це означає, що можна знайти стратегію,
яка при будь-яких значеннях невідомих параметрів буде давати менший ризик,
тобто буде менше помилятися.
