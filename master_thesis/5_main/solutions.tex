\section{Розв'язок}

\subsection{Інтервальна функція витрат}

Розглянемо функцію витрат,
за якої вірним є будь-який набір параметрів,
що знаходиться в $\delta$-околі дійсного набору $x$,
а за всі інші сплачується штраф $1$.

Дискретизуємо простір параметрів $X$.
Введемо невелике $\delta x$ таке,
щоб ймовірність того, що значення потрапляє в рамки гіперкубу
$\left[ x - \frac{\delta x}{2}; x + \frac{\delta x}{2} \right]$,
приблизно дорівнювала добутку щільності розподілу в його центрі
на його об'єм
\begin{equation*}
  \mathbb{P}\left( \xi \in \left[ x - \frac{\delta x}{2};
                                  x + \frac{\delta x}{2} \right] \right)
  \approx p\left( x \right) \cdot \delta x^n,
\end{equation*}
де
\begin{equation*}
  \left( x \pm \frac{\delta x}{2} \right)_i = x_i \pm \frac{\delta x}{2},\qquad
  i = 1..n.
\end{equation*}
Далі треба розбити простір на гіперкуби
зі стороною $\delta x$ і вважати,
що коли величина потрапляє на територію певного гіперкубу,
то вона дорівнює значенню в його центрі.

Відомо, що параметри $x$ --- набори коефіцієнтів нормованих головних компонент,
тобто мають стандартний гаусовий розподіл.
Ймовірність того, що було згенеровано саме такий набір
\begin{equation*}
  \mathbb{P}_X\left( x \right)
  = \delta x^n \cdot \prod_{i=1}^n
    \frac{\exp{\left( - \frac{x_i^2}{2} \right)}}{\sqrt{2 \cdot \pi}}.
\end{equation*}

Ймовірність того, що обличчя на даному зображенні отримано з параметрами $x$
\begin{equation*}
  \mathbb{P}_I\left( t \;\middle|\; x \right)
  = \delta x^{w \cdot h} \cdot \prod_{i \in I}
    \frac{\exp{\left\{- \frac{\left( t_i - f_i\left( x \right) \right)^2}
           {2 \cdot \sigma^2_t} \right\}}}
           {\sqrt{2 \cdot \pi \cdot \sigma^2_t}}.
\end{equation*}

Для максимізації розглянемо логарифми ймовірностей,
тому що це зручніше,
ніж максимізація добутку кількох десятків тисяч або мільйонів значень
\begin{equation*}
  \begin{split}
    \mathbb{P}_I\left( t \;\middle|\; x \right)
    &= \ln{\delta x^{w \cdot h}} +
        \sum_{i \in I}
        \left\{
          - \frac{\left( t_i - f_i\left( x \right) \right)^2}{2 \cdot \sigma^2}
          - \frac{\ln{2} + \ln{\pi} + 2 \cdot \ln{\sigma}}{2}
        \right\}, \\
    \mathbb{P}_X\left( x \right)
    &= \ln{\delta x^n} +
        \sum_{i = 1}^n
        \left\{
          - \frac{x_i^2}{2}
          - \frac{\ln{2} + \ln{\pi}}{2}
        \right\}.
  \end{split}
\end{equation*}
Приберемо константні доданки,
що не впливають на результат максимізації ймовірності
появи даного зображення $t$ з даними параметрами $x$
\begin{equation*}
  \sum_{i \in I}
    \left\{
      - \frac{\left( t_i - f_i\left( x \right) \right)^2}{2 \cdot \sigma^2}
      - \frac{\ln{\sigma^2}}{2}
    \right\}
  - \sum_{i = 1}^n \frac{x_i^2}{2}
  \to \max\limits_{x \in X}.
\end{equation*}
Помножимо на $-2$ та винесемо логарифм дисперсії за знак суми
\begin{equation}\label{eq:minimize}
  w \cdot h \cdot \ln{\sigma^2}
  + \sum_{i \in I} \frac{\left( t_i - f_i\left( x \right) \right)^2}{\sigma^2}
  + \sum_{i = 1}^n x_i^2
  \to \min\limits_{x \in X}.
\end{equation}

Відомо, що $n$ набагато менше за $w \cdot h$.
Визначимо, за яких умов можна знехтувати ймовірністю набору параметрів $x$.
Для деякого $\varepsilon$ повинна виконуватись нерівність
\begin{equation*}
  \left|
    \frac{\sum\limits_{i = 1}^n x_i^2}
         {w \cdot h \cdot \ln{\sigma^2}
          + \sum\limits_{i \in I}
            \frac{\left( t_i - f_i\left( x \right) \right)^2}{\sigma^2}}
  \right|
  < \varepsilon.
\end{equation*}
Розв'яжемо цю нерівність відносно дисперсії шуму
\begin{equation*}
  \frac{\sum\limits_{i = 1}^n x_i^2}{\varepsilon \cdot w \cdot h}
  < \left| \ln\sigma^2
    + \frac{\sum\limits_{i \in I} \left( t_i - f_i\left( x \right) \right)^2}
           {\sigma^2 \cdot w \cdot h}
   \right|.
\end{equation*}
Права частина містить суму, яка дорівнює оцінці дисперсії
діленій на реальну дисперсію шума.
Вважаємо, що оцінка має достатньо елементів,
щоб сума була приблизно рівною одиниці
\begin{equation*}
  \frac{\sum\limits_{i = 1}^n x_i^2}{\varepsilon \cdot w \cdot h}
  < \left| \ln\sigma^2 + 1 \right|.
\end{equation*}
Потрібно позбутися модуля.
Якщо $\ln\sigma^2 + 1 > 0$, то
\begin{equation*}
  \sigma^2 > e^{-1} \approx 0.37.
\end{equation*}
Це завеликий шум, який сильно спотворить зображення,
тому не будемо розглядати цей випадок.
Отримали нерівність
\begin{equation*}
  - \frac{\sum\limits_{i = 1}^n x_i^2}{\varepsilon \cdot w \cdot h} - 1
  > \ln\sigma^2.
\end{equation*}
Візьмемо експоненту від обох частин
\begin{equation*}
  \sigma^2
  < \exp{\left\{
      - \frac{\sum\limits_{i = 1}^n x_i^2}
             {\varepsilon \cdot w \cdot h}
      - 1
    \right\}}.
\end{equation*}
Перепишемо суму квадратів параметрів моделі як їх середнє помножене на кількість
\begin{equation*}
  \sigma^2
  < \exp{\left\{
      - \frac{n \cdot \overline{\left\| x \right\|^2}}
             {\varepsilon \cdot w \cdot h}
      - 1
    \right\}}.
\end{equation*}
Для зображеннь,
що використовуються на практиці,
дріб буде приймати настільки мале значення,
що їм можна знехтувати
\begin{equation*}
  \sigma^2 < e^{- 1}.
\end{equation*}
Це означає, що внеском ймовірності параметрів $x$ теж можна знехтувати
на не зовсім малих зображеннях (наприклад, $500 \times 500$).

Повернемося до \eqref{eq:minimize} та замінимо дисперсію на її оцінку
\begin{equation*}
  w \cdot h
  \cdot \ln{
    \frac{\sum\limits_{i \in I}\left( t_i - f_i\left( x \right) \right)^2}
         {w \cdot h - 1}}
  + w \cdot h - 1
  + \sum_{i = 1}^n x_i^2
  \to \min\limits_{x \in X}.
\end{equation*}
Позбавимося константних доданків та поділимо все на розмір зображення
\begin{equation*}
  \ln{\sum_{i \in I} \left( t_i - f_i\left( x \right) \right)^2}
  + \frac{\sum\limits_{i = 1}^n x_i^2}{w \cdot h}
  \to \min\limits_{x \in X}.
\end{equation*}
В загальному випадку зображення містить не тільки обличчя.
Генерація усіх можливих фонів для кожної моделі є недоцільною.
Потрібно отримати формулу,
яка не прив'язана до кожного пікселя вхідного зображення $t$.
Для цього віднімемо логарифм розміру зображення.
Це константа, тому не вплине на кінцевий розв'язок
\begin{equation*}
  \ln{\frac{\sum\limits_{i \in I} \left( t_i - f_i\left( x \right) \right)^2}
           {w \cdot h}}
  + \frac{\sum\limits_{i = 1}^n x_i^2}{w \cdot h}
  \to \min\limits_{x \in X}.
\end{equation*}
Перепишемо через середнє
\begin{equation*}
  w \cdot h \cdot \ln{\overline{\left\| t - f\left( x \right) \right\|^2}}
  + n \cdot \overline{\left\| x \right\|^2}
  \to \min\limits_{x \in X}.
\end{equation*}
Можна брати середнє значення шуму по тим пікселям,
де на згенерованому зображенні знаходиться обличчя.
Оскільки вважається, що накладений шум ---
це набір незалежних випадкових однаково розподілених величин,
середнє будь-якої їх сукупності
буде оптимальною оцінкою математичного очікування їх.

Знехтуємо ймовірністю параметрів $x$
\begin{equation*}
  \overline{\left\| t - f\left( x \right) \right\|^2}
  \to \min\limits_{x \in X}.
\end{equation*}
Помножимо на $w \cdot h$.
Отримали мінімізацію відомої цільової функції --- суми квадратів відхилень
\begin{equation*}
  \sum_{i \in I} \left( t_i - f_i\left( x \right) \right)^2
  \to \min\limits_{x \in X}.
\end{equation*}
Мінімізація суми квадратів різниць між дійсним та згенерованим зображенням
розв'язує задачу з бінарною або інтервальною функцією штрафу
та гаусовим шумом на зображенні без додаткових умов
на кшталт розподілу параметрів $x$.

\subsection{Різниця параметрів з урахуванням їх гаусового розподілу}

Вважаємо, що на даному зображенні $t$
присутній нормальний шум з невідомою дисперсією $\sigma^2_t$.
Ймовірність того, що зображення моделі з конкретними параметрами $x$
є очищеним від шумів зображенням $t$
\begin{equation*}
  \mathbb{P}_I\left( x \mid t \right)
  = \delta x^n \cdot \prod_{i \in I}
    \frac{\exp{\left\{ - \frac{\left( t_i - f_i\left( x \right) \right)^2}
    {2 \cdot \sigma_t^2} \right\}}}{\sqrt{2 \cdot \pi \cdot \sigma^2}}.
\end{equation*}
Відомо, що параметри $x$ --- набори коефіцієнтів нормованих головних компонент,
тобто мають стандартний гаусовий розподіл.
Ймовірність того, що було згенеровано саме такий набір
\begin{equation*}
  \mathbb{P}_X\left( x \mid t \right)
  = \delta x^n \cdot \prod_{i=1}^n
    \frac{\exp{\left( - \frac{x_i^2}{2} \right)}}{\sqrt{2 \cdot \pi}}.
\end{equation*}
Ймовірність того,
що дане зображення $t$ було отримано саме з параметрами $x$,
є добутком розглянутих ймовірностей
\begin{equation*}
  \mathbb{P}\left( x \mid t \right)
  = \mathbb{P}_I\left( x \mid t \right)
    \cdot \mathbb{P}_X\left( x \mid t \right)
    = \delta x^{2 \cdot n} \cdot \prod_{i \in I}
    \frac{\exp{\left\{ - \frac{\left( t_i - f_i\left( x \right) \right)^2}
         {2 \cdot \sigma_t^2} \right\}}}{\sqrt{2 \cdot \pi \cdot \sigma_t^2}}
    \cdot
    \prod_{i=1}^n
    \frac{\exp{\left( - \frac{x_i^2}{2} \right)}}{\sqrt{2 \cdot \pi}}.
\end{equation*}
Коли з контексту буде зрозуміло,
якому саме зображенню належить дана дисперсія,
індекс $t$ не будемо використовувати.

Винесемо константи з добутків та позначимо
\begin{equation*}
  c_t = \left( 2 \cdot \pi \right)^{-\frac{w \cdot h + n}{2}}
    \cdot \sigma_t^{- w \cdot h}.
\end{equation*}
Ймовірність приймає вигляд
\begin{equation*}
  \mathbb{P}\left( x \mid t \right)
  = c_t \cdot \delta x^{2 \cdot n}
    \cdot \exp{\left\{ - \frac{\left\| t - f\left( x \right) \right\|^2}
                              {2 \cdot \sigma^2} \right\}}
    \cdot \exp{\left\{ - \frac{\left\| x \right\|^2}{2} \right\}}.
\end{equation*}
Стратегія розпізнавання
\begin{equation*}
  q^* \left( t \right)
  = c_t \cdot \delta x^{2 \cdot n}
    \cdot \sum_{x \in X}
      x
      \cdot \exp{\left\{ - \frac{\left\| t - f\left( x \right) \right\|^2}
                                {2 \cdot \sigma^2} \right\}}
    \cdot \exp{\left\{ - \frac{\left\| x \right\|^2}{2} \right\}}.
\end{equation*}
У неперервному випадку сума змінюється на інтеграл та зникають об'єми гіперкубів
\begin{equation*}
  q^* \left( t \right)
  = c_t
    \cdot \int\limits_{x \in X}
      x
      \cdot \exp{\left\{ - \frac{\left\| t - f\left( x \right) \right\|^2}
                                {2 \cdot \sigma^2} \right\}}
      \cdot \exp{\left\{ - \frac{\left\| x \right\|^2}{2} \right\}}
    d\,x.
\end{equation*}

Оскільки $\sigma^2$ невідома,
її можна оцінити за відомою формулою оптимальної оцінки дисперсії
\begin{equation*}
  \overline{\sigma^2}
  = \sum_{x \in X}
    \frac{\left\| f\left( x \right) - t \right\|^2}{w \cdot h \cdot n - 1}.
\end{equation*}
