\section{Розв'язок}

\subsection{Інтервальна функція витрат}

Розглянемо функцію витрат,
за якої вірним є будь-який набір параметрів,
що знаходиться в $\delta$-околі дійсного набору $x$,
а за всі інші сплачується штраф $1$.

Дискретизуємо простір параметрів $X$.
Введемо невелике $\delta x$ таке,
щоб ймовірність того, що значення потрапляє в рамки гіперкубу
$\left[ x - \frac{\delta x}{2}; x + \frac{\delta x}{2} \right]$,
приблизно дорівнювала добутку щільності розподілу в його центрі
на його об'єм
\begin{equation*}
  \mathbb{P}\left( \xi \in \left[ x - \frac{\delta x}{2};
                                  x + \frac{\delta x}{2} \right] \right)
  \approx p\left( x \right) \cdot \delta x^n,
\end{equation*}
де
\begin{equation*}
  \left( x \pm \frac{\delta x}{2} \right)_i = x_i \pm \frac{\delta x}{2},\qquad
  i = 1..n.
\end{equation*}
Далі треба розбити простір на гіперкуби
зі стороною $\delta x$ і вважати,
що коли величина потрапляє на територію певного гіперкубу,
то вона дорівнює значенню в його центрі.
Позначимо
\begin{equation*}
  \mathbb{P}_I\left( x \;\middle|\; t \right)
  = \mathbb{P}_I\left( \xi = x \;\middle|\; t \right)
  = \mathbb{P}\left( \xi \in \left[ x - \frac{\delta x}{2};
                                    x + \frac{\delta x}{2} \right]
    \;\middle|\; t \right)
  \approx p\left( x \;\middle|\; t \right) \cdot \delta x^n.
\end{equation*}

Вважаємо, що на даному зображенні $t$
присутній нормальний шум з невідомою дисперсією $\sigma^2_t$.
Тоді ймовірність того,
що дане зображення було отримано саме з параметрами $x$
\begin{equation*}
  \mathbb{P}_I\left( x \;\middle|\; t \right)
  = \delta x^n \cdot \prod_{i \in I}
    \frac{\exp{\left\{- \frac{\left( t_i - f_i\left( x \right) \right)^2}
           {2 \cdot \sigma^2_t} \right\}}}
           {\sqrt{2 \cdot \pi \cdot \sigma^2_t}}.
\end{equation*}
Коли з контексту буде зрозуміло,
якому саме зображенню належить дана дисперсія,
індекс $t$ не будемо використовувати.

Максимізуємо логарифм ймовірності,
тому що це зручніше,
ніж максимізація добутку кількох десятків тисяч або мільйонів значень
\begin{equation*}
  \ln{\mathbb{P}_I\left( x \mid t \right)}
  = \ln{\delta x^n} + \sum_{i \in I}
    \left\{
      - \frac{\left( t_i - f_i\left( x \right) \right)^2}{2 \cdot \sigma^2}
      - \frac{\ln{2} + \ln{\pi} + 2 \cdot \ln{\sigma}}{2}
    \right\}
  \to \max.
\end{equation*}
Позбуваємося константного доданку та помножимо на подвоєну ненульову дисперсію
\begin{equation*}
  \sum_{i \in I} \left( t_i - f_i\left( x \right) \right)^2 \to \min.
\end{equation*}
Бачимо,
що мінімізація суми квадратів різниць між дійсним та згенерованим зображенням
розв'язує задачу з бінарною функцією втрат
та гаусовим шумом на зображенні без інших додаткових умов,
які буде розглянуто далі.

Введемо множину пікселів $F \subset I$,
які на зображенні $f\left( x \right)$ належать обличчю.
Тоді суму можно розбити на дві
\begin{equation*}
  \sum_{i \in I} \left( t_i - f_i\left( x \right) \right)^2
  = \sum_{i \in F} \left( t_i - f_i\left( x \right) \right)^2
  + \sum_{i \in I \setminus F} \left( t_i - f_i\left( x \right) \right)^2.
\end{equation*}
Рахуємо вибірковую дисперсію для тієї частини зображення,
де зображено обличчя
\begin{equation*}
  \overline{\sigma_F^2}
  = \frac{\sum\limits_{i \in F} \left( t_i - f_i\left( x \right) \right)^2}
    {\left| F - 1 \right|}.
\end{equation*}
Екстраполюємо це значення на все зображення
\begin{equation*}
  \overline{\sigma_F^2}
  \approx \frac{\sum\limits_{i \in I}
             \left( t_i - f_i\left( x \right) \right)^2}
            {\left| I \right| - 1}.
\end{equation*}
Тоді вираз, який треба мінімізувати, можна представити як
\begin{equation*}
  \sum_{i \in I} \left( t_i - f_i\left( x \right) \right)^2
  \approx \left( \left| I \right| - 1 \right) \cdot \overline{\sigma_F^2}
  \to \min.
\end{equation*}
Оскільки розмір зображення фіксований і його зменшити не вийде,
потрібно зменшити вибіркову дисперсію на пікселях обличчя
\begin{equation*}
  \overline{\sigma_F^2}
  = \frac{\sum\limits_{i \in F} \left( t_i - f_i\left( x \right) \right)^2}
         {\left| F - 1 \right|}
  \to \min.
\end{equation*}

\subsection{Різниця параметрів з урахуванням їх гаусового розподілу}

Вважаємо, що на даному зображенні $t$
присутній нормальний шум з невідомою дисперсією $\sigma^2_t$.
Ймовірність того, що зображення моделі з конкретними параметрами $x$
є очищеним від шумів зображенням $t$
\begin{equation*}
  \mathbb{P}_I\left( x \mid t \right)
  = \delta x^n \cdot \prod_{i \in I}
    \frac{\exp{\left\{ - \frac{\left( t_i - f_i\left( x \right) \right)^2}
    {2 \cdot \sigma_t^2} \right\}}}{\sqrt{2 \cdot \pi \cdot \sigma^2}}.
\end{equation*}
Відомо, що параметри $x$ --- набори коефіцієнтів нормованих головних компонент,
тобто мають стандартний гаусовий розподіл.
Ймовірність того, що було згенеровано саме такий набір
\begin{equation*}
  \mathbb{P}_X\left( x \mid t \right)
  = \delta x^n \cdot \prod_{i=1}^n
    \frac{\exp{\left( - \frac{x_i^2}{2} \right)}}{\sqrt{2 \cdot \pi}}.
\end{equation*}
Ймовірність того,
що дане зображення $t$ було отримано саме з параметрами $x$,
є добутком розглянутих ймовірностей
\begin{equation*}
  \mathbb{P}\left( x \mid t \right)
  = \mathbb{P}_I\left( x \mid t \right)
    \cdot \mathbb{P}_X\left( x \mid t \right)
    = \delta x^{2 \cdot n} \cdot \prod_{i \in I}
    \frac{\exp{\left\{ - \frac{\left( t_i - f_i\left( x \right) \right)^2}
         {2 \cdot \sigma_t^2} \right\}}}{\sqrt{2 \cdot \pi \cdot \sigma_t^2}}
    \cdot
    \prod_{i=1}^n
    \frac{\exp{\left( - \frac{x_i^2}{2} \right)}}{\sqrt{2 \cdot \pi}}.
\end{equation*}
Коли з контексту буде зрозуміло,
якому саме зображенню належить дана дисперсія,
індекс $t$ не будемо використовувати.

Винесемо константи з добутків та позначимо
\begin{equation*}
  c_t = \left( 2 \cdot \pi \right)^{-\frac{w \cdot h + n}{2}}
    \cdot \sigma_t^{- w \cdot h}.
\end{equation*}
Ймовірність приймає вигляд
\begin{equation*}
  \mathbb{P}\left( x \mid t \right)
  = c_t \cdot \delta x^{2 \cdot n}
    \cdot \exp{\left\{ - \frac{\left\| t - f\left( x \right) \right\|^2}
                              {2 \cdot \sigma^2} \right\}}
    \cdot \exp{\left\{ - \frac{\left\| x \right\|^2}{2} \right\}}.
\end{equation*}
Стратегія розпізнавання
\begin{equation*}
  q^* \left( t \right)
  = c_t \cdot \delta x^{2 \cdot n}
    \cdot \sum_{x \in X}
      x
      \cdot \exp{\left\{ - \frac{\left\| t - f\left( x \right) \right\|^2}
                                {2 \cdot \sigma^2} \right\}}
    \cdot \exp{\left\{ - \frac{\left\| x \right\|^2}{2} \right\}}.
\end{equation*}
У неперервному випадку сума змінюється на інтеграл та зникають об'єми гіперкубів
\begin{equation*}
  q^* \left( t \right)
  = c_t
    \cdot \int\limits_{x \in X}
      x
      \cdot \exp{\left\{ - \frac{\left\| t - f\left( x \right) \right\|^2}
                                {2 \cdot \sigma^2} \right\}}
      \cdot \exp{\left\{ - \frac{\left\| x \right\|^2}{2} \right\}}
    d\,x.
\end{equation*}

Оскільки $\sigma^2$ невідома,
її можна оцінити за відомою формулою оптимальної оцінки дисперсії
\begin{equation*}
  \overline{\sigma^2}
  = \sum_{x \in X}
    \frac{\left\| f\left( x \right) - t \right\|^2}{w \cdot h \cdot n - 1}.
\end{equation*}
